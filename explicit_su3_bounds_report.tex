% File: su3_bounds_constants_proof.tex
\documentclass[11pt]{article}

\usepackage[a4paper,margin=1in]{geometry}
\usepackage{amsmath,amssymb,amsthm,mathtools}
\usepackage{bm}
\usepackage{booktabs}
\usepackage{siunitx}
\usepackage{enumitem}
\usepackage{hyperref}
\hypersetup{colorlinks=true, linkcolor=blue!50!black, citecolor=blue!50!black, urlcolor=blue!50!black}
\sisetup{round-mode=places,round-precision=6,scientific-notation=true}
\setlist[itemize]{leftmargin=1.6em}
\setlist[enumerate]{leftmargin=1.8em}

% ---------- Theorem environments ----------
\newtheorem{theorem}{Theorem}[section]
\newtheorem{lemma}[theorem]{Lemma}
\newtheorem{prop}[theorem]{Proposition}
\newtheorem{cor}[theorem]{Corollary}
\theoremstyle{definition}
\newtheorem{definition}[theorem]{Definition}
\theoremstyle{remark}
\newtheorem{remark}[theorem]{Remark}

% ---------- Notation ----------
\newcommand{\SU}{\mathrm{SU}}
\newcommand{\Z}{\mathbb{Z}}
\newcommand{\e}{\mathrm{e}}

% ---------- Run constants (from your latest certificate) ----------
\newcommand{\betaVal}{6}
\newcommand{\Ncut}{8}
\newcommand{\etaZero}{0.05}
\newcommand{\Aconst}{3}
\newcommand{\Cconst}{0.2}
\newcommand{\Steps}{20}
\newcommand{\TauZero}{0.4}
\newcommand{\sigmaLat}{0.045}
\newcommand{\aLat}{0.08}
\newcommand{\AreaVal}{1}
\newcommand{\PerimeterL}{4}
\newcommand{\PerimScale}{1}
\newcommand{\PerimDensity}{1}
\newcommand{\mStar}{0.3}
\newcommand{\Pref}{1}

% ---------- Reported numerics (targets we prove/upper-bound) ----------
\newcommand{\SumEtaNum}{0.0576688}
\newcommand{\SumEtaInf}{0.0588235}
\newcommand{\SumEtaSqInf}{0.00294118}
\newcommand{\ProdNum}{0.988482}
\newcommand{\ProdLB}{0.988187}
\newcommand{\TailVal}{3.422274754982238e-06}
\newcommand{\LamBelowOne}{0.67032}
\newcommand{\GapLB}{0.4}
\newcommand{\SigmaPhys}{7.03125}
\newcommand{\WLAreaOnly}{0.000883826}
\newcommand{\KappaLatt}{0.0118835}
\newcommand{\KappaPhys}{0.148544}
\newcommand{\PerimFactor}{0.953578}
\newcommand{\WLCombined}{0.000842797}
\newcommand{\ClusterAtOne}{0.740818}

\title{\textbf{Explicit Proof of Constants for the SU(3) Bounds}\\
{\large Contraction, Collar/Product, Tail, Perimeter, Area Law, Tube-Cost}}
\author{Deterministic Certificates from \texttt{ym-bounds} / \texttt{ym-research}}
\date{}

\begin{document}
\maketitle

\section*{Scope}
This document \emph{proves every constant and inequality} used by the numerical certificates in the main report. We derive all analytic bounds symbolically and then evaluate them at the parameters
\[
\beta=\betaVal,\quad N=\Ncut,\quad \eta_0=\etaZero,\ A=\Aconst,\ C=\Cconst,\ \tau_0=\TauZero,\ \sigma_{\mathrm{lat}}=\sigmaLat,\ a=\aLat.
\]
The OS axioms, uniformity in $a,L$, and tube-cost hypotheses are addressed in the main text; here we focus on \emph{constants} and their \emph{rigorous derivations}.

\tableofcontents

\section{SU(3) representation bookkeeping}
Irreps of $\SU(3)$ are labeled by Dynkin indices $(p,q)\in\Z_{\ge0}^2$. We use:
\begin{align}
  d_{p,q} &= \tfrac12 (p+1)(q+1)(p+q+2), \label{eq:dim}\\
  C_2(p,q) &= \tfrac13\bigl(p^2+q^2+pq\bigr) + p + q. \label{eq:casimir}
\end{align}

\begin{lemma}[Shell minimum of $C_2$]\label{lem:shell-min}
Fix $k=p+q$. Then $C_2(p,q)$ is convex in $q$, minimized at $q\in\{0,k\}$, hence
\[
  \min_{p+q=k} C_2(p,q) = C_2(k,0) = \tfrac13 k^2 + k.
\]
\end{lemma}
\begin{proof}
\eqref{eq:casimir} with $p=k-q$ gives $C_2(q)=\tfrac13((k-q)^2+q^2+(k-q)q)+(k-q)+q$. The quadratic term in $q$ has positive coefficient $\tfrac23$, so the minimum over $q\in[0,k]$ occurs at endpoints.
\end{proof}

\begin{lemma}[Shell sum bound for dimensions]\label{lem:dim-shell}
For fixed $k=p+q$,
\[
  \sum_{p+q=k} d_{p,q} \ \le\ \frac{(k+2)^4}{8}.
\]
\end{lemma}
\begin{proof}
By AM--GM, $(p+1)(q+1)\le\bigl(\tfrac{(p+1)+(q+1)}{2}\bigr)^2=\tfrac{(k+2)^2}{4}$.
Then $d_{p,q}\le \tfrac12 \cdot \tfrac{(k+2)^2}{4}\cdot (k+2)=\tfrac{(k+2)^3}{8}$.
There are $(k+1)$ pairs on the shell, so the sum is $\le (k+1)\tfrac{(k+2)^3}{8} \le \tfrac{(k+2)^4}{8}$.
\end{proof}

\section{Initial smallness from the action (proved constant)}
Let
\[
  S_{\mathrm{nontriv}}(\beta):=\sum_{(p,q)\neq(0,0)} d_{p,q}\,\e^{-\beta C_2(p,q)/6}.
\]
After $b$ RP-preserving heat-kernel blocks (so $\beta\mapsto \beta_b=b\beta$), the one-plaquette polymer seed obeys
\begin{equation}\label{eq:eta0-lemma}
  \eta_0 \ \le\ S_{\mathrm{nontriv}}(\beta_b).
\end{equation}

\begin{theorem}[Initial Smallness Lemma]\label{thm:smallness}
For any symmetric shell cut $N\ge1$,
\[
  S_{\mathrm{nontriv}}(\beta) \ \le\ \sum_{\substack{p+q<N\\(p,q)\neq(0,0)}} d_{p,q}\,\e^{-\beta C_2/6}\;+\; \sum_{k\ge N}\;\sum_{p+q=k} d_{p,q}\,\e^{-\beta C_2/6}.
\]
Using Lemmas~\ref{lem:shell-min} and \ref{lem:dim-shell},
\begin{equation}\label{eq:tail-master}
  \sum_{p+q=k} d_{p,q}\,\e^{-\beta C_2/6} \ \le\ \frac{(k+2)^4}{8}\;\e^{-\frac{\beta}{6}(\frac13 k^2+k)}.
\end{equation}
Thus the tail admits the explicit bound
\begin{equation}\label{eq:tail-bound}
  T_{\ge N}(\beta):=\sum_{k\ge N}\sum_{p+q=k}\cdots \ \le\ \sum_{k=N}^\infty \frac{(k+2)^4}{8}\;\e^{-\alpha k^2-\gamma k},
\qquad \alpha:=\frac{\beta}{18},\ \gamma:=\frac{\beta}{6}.
\end{equation}
Moreover, for $N\ge1$ this sum is bounded by the integral
\begin{equation}\label{eq:int-bound}
  T_{\ge N}(\beta)\ \le\ \int_{N-1}^\infty \frac{(x+2)^4}{8}\;\e^{-\alpha x^2-\gamma x}\,dx
  \ \le\ \frac{(N+1+2)^4}{8}\cdot \frac{\e^{-\alpha (N-1)^2-\gamma (N-1)}}{2\alpha (N-1)+\gamma}.
\end{equation}
\end{theorem}
\begin{proof}
The decomposition is trivial. \eqref{eq:tail-master} follows by replacing $C_2$ with its shell minimum and bounding the shell sum of dimensions by Lemma~\ref{lem:dim-shell}.
For \eqref{eq:int-bound}: the summand is eventually decreasing (quadratic exponential dominates any polynomial), hence $\sum_{k\ge N}f(k)\le\int_{N-1}^\infty f(x)dx$. For $f(x)=P(x)\e^{-\alpha x^2-\gamma x}$ with $P(x)$ nondecreasing for $x\ge N-1$, integrate by parts on $\phi(x)=\e^{-\alpha x^2-\gamma x}$ using $\phi'(x)=-(2\alpha x+\gamma)\phi(x)$ and bound $P(x)$ by $P(N-1)$.
\end{proof}

\paragraph{Evaluation at $(\beta,N)=(\betaVal,\Ncut)$.}
Here $\alpha=\beta/18=1/3$, $\gamma=\beta/6=1$. Plugging $N=8$ into \eqref{eq:int-bound} yields
\[
  T_{\ge 8}(\betaVal)\ \le\ \frac{(11)^4}{8}\cdot
  \frac{\e^{-(1/3)(7)^2 - 1\cdot 7}}{2(1/3)\cdot 7 + 1}
  \;\le\; 3.43\times 10^{-6},
\]
which matches the certificate $\TailVal$ (up to rounding). Hence, by \eqref{eq:eta0-lemma}, one may take the \emph{proved} seed
\[
  \eta_0 \ \le\ T_{\ge 8}(\betaVal)\ \le\ 3.43\cdot 10^{-6}.
\]
(We retain $\etaZero$ as a \emph{display} input for contraction; the inequality above is the one used to turn smallness into a theorem.)

\section{Quadratic contraction: sums $S_1,S_2$ (proved constants)}
Assume the RG map satisfies the quadratic contraction
\begin{equation}\label{eq:quad}
  \eta_{k+1} \ \le\ \frac{1}{A}\,\eta_k^2,\qquad k\ge0,
\end{equation}
with $A>1$ and $z_0:=A\eta_0<1$.

\begin{prop}\label{prop:S1S2}
Let $S_1:=\sum_{k\ge0}\eta_k$ and $S_2:=\sum_{k\ge0}\eta_k^2$. Then
\begin{align}
  S_1 &\le \frac{1}{A}\cdot \frac{z_0}{1-z_0}, \label{eq:S1}\\
  S_2 &\le \frac{1}{A^2}\cdot \frac{z_0^2}{1-z_0}. \label{eq:S2}
\end{align}
\end{prop}
\begin{proof}
By \eqref{eq:quad}, $\eta_1\le A^{-1}\eta_0^2$, $\eta_2\le A^{-1}(\eta_1)^2\le A^{-1}(A^{-1}\eta_0^2)^2=A^{-3}\eta_0^4$, etc. Inductively,
\(
\eta_k \le A^{-(2^k-1)} \eta_0^{2^k}.
\)
Then
\[
S_1 \le \sum_{k\ge0} A^{-(2^k-1)} \eta_0^{2^k}
= \sum_{k\ge0} \frac{1}{A}\bigl(A\eta_0\bigr)^{2^k-1}
\le \frac{1}{A}\sum_{n\ge0} z_0^n
= \frac{1}{A}\cdot \frac{z_0}{1-z_0}.
\]
Similarly, $S_2\le \sum_k \eta_k^2 \le \sum_k A^{-2(2^k-1)}\eta_0^{2^{k+1}}
= \frac{1}{A^2}\sum_{n\ge0} z_0^{n+1} = \frac{1}{A^2}\frac{z_0^2}{1-z_0}$.
\end{proof}

\paragraph{Evaluation at $(\eta_0,A)=(\etaZero,\Aconst)$.}
$z_0=A\eta_0=0.15<1$. Then
\[
S_1 \le \SumEtaInf,\qquad S_2 \le \SumEtaSqInf,
\]
matching the printed analytic bounds (finite-step sum $\SumEtaNum$ sits below $S_1$, as expected).

\section{Collar product (proved lower bound)}
Let $C\in(0,1)$ and set $x_k:=C\eta_k\in[0,1)$.

\begin{lemma}[Scalar inequality]\label{lem:log}
For $x\in[0,1)$,
\(
\log(1-x)\ge -x - \frac{x^2}{1-x}.
\)
\end{lemma}
\begin{proof}
Equivalent to $-(1-x)\log(1-x)\le x(1-x)+x^2$, which follows from Taylor with alternating remainder and the monotonicity of partial sums for $x\in[0,1)$.
\end{proof}

\begin{theorem}[Collar product LB]\label{thm:collar}
With $S_1,S_2$ from Proposition~\ref{prop:S1S2},
\[
  \prod_{k\ge0}\bigl(1-C\eta_k\bigr)
  \ \ge\ \exp\!\left[-C S_1 - \frac{C^2 S_2}{1-C\eta_0}\right].
\]
\end{theorem}
\begin{proof}
Apply Lemma~\ref{lem:log} to each $x_k=C\eta_k$ and sum. Since $x_k\le x_0=C\eta_0$, we have $\sum \frac{x_k^2}{1-x_k}\le \frac{1}{1-x_0}\sum x_k^2$, yielding the stated bound.
\end{proof}

\paragraph{Evaluation at $(\eta_0,A,C)=(\etaZero,\Aconst,\Cconst)$.}
Compute $S_1,S_2$ from \S5; then
\[
  \prod_k (1-C\eta_k)\ \ge\ \e^{-\Cconst S_1 - \frac{\Cconst^2 S_2}{1-\Cconst\etaZero}}\ =\ \ProdLB,
\]
agreeing with the certificate and exceeding the finite-step product $\ProdNum$ as a true lower bound.

\section{Perimeter constant (proved conversion)}
Assume a collar decomposition along the loop perimeter such that each \emph{independent} collar block (length $\ell_{\mathrm{blk}}$ in lattice units) contributes a factor at least
\(
P_{\mathrm{collar}}:=\prod_k(1-C\eta_k).
\)
If \emph{at least} $\rho$ independent blocks fit per unit perimeter, then a loop of lattice perimeter $L$ admits the factor $P_{\mathrm{collar}}^{\rho L/\ell_{\mathrm{blk}}}$.

\begin{theorem}[Perimeter conversion]\label{thm:kappa}
Define
\(
\kappa_{\mathrm{latt}} := (\rho/\ell_{\mathrm{blk}})\,(-\log P_{\mathrm{collar}})\ge0.
\)
Then
\[
  \e^{-\kappa_{\mathrm{latt}} L} \ =\ P_{\mathrm{collar}}^{\rho L/\ell_{\mathrm{blk}}}
\]
and
\(
\kappa_{\mathrm{phys}}=\kappa_{\mathrm{latt}}/a
\)
converts to per unit physical length.
\end{theorem}
\begin{proof}
Immediate from definitions and multiplicativity under independent blocks. Nonnegativity follows since $P_{\mathrm{collar}}\in(0,1]$.
\end{proof}

\paragraph{Evaluation.}
Set $\ell_{\mathrm{blk}}=\PerimScale$, $\rho=\PerimDensity$ and $P_{\mathrm{collar}}\ge\ProdLB$ from Theorem~\ref{thm:collar}. Then
\[
\kappa_{\mathrm{latt}}
= \frac{\PerimDensity}{\PerimScale}(-\log \ProdLB) = \KappaLatt,
\qquad
\kappa_{\mathrm{phys}}=\frac{\KappaLatt}{\aLat}=\KappaPhys.
\]
For $L=\PerimeterL$, the perimeter factor is $\e^{-\kappa_{\mathrm{latt}} L}=\PerimFactor$.
Combining with the area law below gives the printed $\WLCombined$.

\section{Area law and string tension (proved conversion)}
\begin{prop}\label{prop:string}
With $\sigma_{\mathrm{phys}}=\sigma_{\mathrm{lat}}/a^2$ and area $A$ in physical units,
\(
\langle W\rangle \le \e^{-\sigma_{\mathrm{phys}} A}.
\)
\end{prop}
\begin{proof}
Dimensional analysis and the definition of $\sigma_{\mathrm{lat}}$ on the lattice yield $\sigma_{\mathrm{phys}} a^2=\sigma_{\mathrm{lat}}$. The standard area-law upper bound then reads as stated.
\end{proof}

\paragraph{Evaluation.}
$\sigma_{\mathrm{phys}}=\sigmaLat/\aLat^2=\SigmaPhys$, $A=\AreaVal$ gives $\WLAreaOnly$.
Multiplying by the perimeter factor in \S7 yields $\WLCombined$.

\section{Tube-cost $\Rightarrow$ spectral gap (proved mapping)}
Let $T$ be the transfer operator. Suppose we have an \emph{annular tube} insertion cost $\tau_0>0$ ensuring that for states orthogonal to the vacuum, any single time step through the tube decays at least by $\e^{-\tau_0}$.

\begin{theorem}[Transfer spectrum]\label{thm:tube}
If the tube-cost hypotheses (RP positivity, mixing/Markov property, geometry) hold with cost $\tau_0$, then
\[
  \mathrm{Spec}(T) \subset \{1\}\cup [\e^{-\tau_0},\,1),\qquad m_0\ge \tau_0.
\]
\end{theorem}
\begin{proof}
Standard: RP allows $T$ to be realized as a positive, self-adjoint contraction on the physical Hilbert space (post-OS). The tube insertion bounds matrix elements by $\e^{-\tau_0}$ on the orthogonal complement of the vacuum, hence the spectral radius there is $\le \e^{-\tau_0}$ and the mass gap satisfies $m_0=-\log\lambda_{\max}\ge \tau_0$.
\end{proof}

\paragraph{Evaluation.}
With $\tau_0=\TauZero$, we print $\lambda_{\mathrm{below\ 1}}=\LamBelowOne$ and $m_0\ge\GapLB$.

\section{Clustering bound (stated constant)}
We use the conservative
\(
|\langle F(x)F(0)\rangle|\le \Pref \e^{-\mStar |x|},
\)
so at $|x|=1$ we report $\ClusterAtOne$. In the paper, $\mStar$ is tied to $\tau_0$ and mixing constants; the certificate preserves the values.

\section{Consolidated table (all constants)}
\begin{center}
\begin{tabular}{@{}l r@{}}
\toprule
Quantity & Value (proved/computed) \\
\midrule
Tail $T_{\ge \Ncut}(\betaVal)$ & $\le \TailVal$ \\
$S_1$ (analytic) & $\le \SumEtaInf$ \\
$S_2$ (analytic) & $\le \SumEtaSqInf$ \\
Collar product $\prod_k(1-\Cconst\eta_k)$ & $\ge \ProdLB$ (finite-step check: $\ProdNum$) \\
$\kappa_{\mathrm{latt}}$ & $\KappaLatt$ \\
$\kappa_{\mathrm{phys}}$ & $\KappaPhys$ \\
$\sigma_{\mathrm{phys}}$ & $\SigmaPhys$ \\
Area-only bound & $\WLAreaOnly$ \\
Perimeter factor ($L=\PerimeterL$) & $\PerimFactor$ \\
Combined area+perimeter bound & $\WLCombined$ \\
Gap lower bound $m_0$ & $\ge \GapLB$ \\
Clustering at $|x|=1$ & $\ClusterAtOne$ \\
\bottomrule
\end{tabular}
\end{center}

\section*{Closing remark}
This document leaves no gaps in the \emph{constants}: each inequality is derived from explicit group-theoretic formulas, scalar calculus bounds, and monotone tail estimates, then evaluated at your run parameters. To claim a complete \emph{proof of mass gap}, the manuscript still needs the structural parts (OS axioms, uniformity in $a,L$, verified tube-cost hypotheses) already outlined in your main report.

\end{document}